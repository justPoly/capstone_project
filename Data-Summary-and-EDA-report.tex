% Options for packages loaded elsewhere
\PassOptionsToPackage{unicode}{hyperref}
\PassOptionsToPackage{hyphens}{url}
%
\documentclass[
]{article}
\usepackage{amsmath,amssymb}
\usepackage{iftex}
\ifPDFTeX
  \usepackage[T1]{fontenc}
  \usepackage[utf8]{inputenc}
  \usepackage{textcomp} % provide euro and other symbols
\else % if luatex or xetex
  \usepackage{unicode-math} % this also loads fontspec
  \defaultfontfeatures{Scale=MatchLowercase}
  \defaultfontfeatures[\rmfamily]{Ligatures=TeX,Scale=1}
\fi
\usepackage{lmodern}
\ifPDFTeX\else
  % xetex/luatex font selection
\fi
% Use upquote if available, for straight quotes in verbatim environments
\IfFileExists{upquote.sty}{\usepackage{upquote}}{}
\IfFileExists{microtype.sty}{% use microtype if available
  \usepackage[]{microtype}
  \UseMicrotypeSet[protrusion]{basicmath} % disable protrusion for tt fonts
}{}
\makeatletter
\@ifundefined{KOMAClassName}{% if non-KOMA class
  \IfFileExists{parskip.sty}{%
    \usepackage{parskip}
  }{% else
    \setlength{\parindent}{0pt}
    \setlength{\parskip}{6pt plus 2pt minus 1pt}}
}{% if KOMA class
  \KOMAoptions{parskip=half}}
\makeatother
\usepackage{xcolor}
\usepackage[margin=1in]{geometry}
\usepackage{color}
\usepackage{fancyvrb}
\newcommand{\VerbBar}{|}
\newcommand{\VERB}{\Verb[commandchars=\\\{\}]}
\DefineVerbatimEnvironment{Highlighting}{Verbatim}{commandchars=\\\{\}}
% Add ',fontsize=\small' for more characters per line
\usepackage{framed}
\definecolor{shadecolor}{RGB}{248,248,248}
\newenvironment{Shaded}{\begin{snugshade}}{\end{snugshade}}
\newcommand{\AlertTok}[1]{\textcolor[rgb]{0.94,0.16,0.16}{#1}}
\newcommand{\AnnotationTok}[1]{\textcolor[rgb]{0.56,0.35,0.01}{\textbf{\textit{#1}}}}
\newcommand{\AttributeTok}[1]{\textcolor[rgb]{0.13,0.29,0.53}{#1}}
\newcommand{\BaseNTok}[1]{\textcolor[rgb]{0.00,0.00,0.81}{#1}}
\newcommand{\BuiltInTok}[1]{#1}
\newcommand{\CharTok}[1]{\textcolor[rgb]{0.31,0.60,0.02}{#1}}
\newcommand{\CommentTok}[1]{\textcolor[rgb]{0.56,0.35,0.01}{\textit{#1}}}
\newcommand{\CommentVarTok}[1]{\textcolor[rgb]{0.56,0.35,0.01}{\textbf{\textit{#1}}}}
\newcommand{\ConstantTok}[1]{\textcolor[rgb]{0.56,0.35,0.01}{#1}}
\newcommand{\ControlFlowTok}[1]{\textcolor[rgb]{0.13,0.29,0.53}{\textbf{#1}}}
\newcommand{\DataTypeTok}[1]{\textcolor[rgb]{0.13,0.29,0.53}{#1}}
\newcommand{\DecValTok}[1]{\textcolor[rgb]{0.00,0.00,0.81}{#1}}
\newcommand{\DocumentationTok}[1]{\textcolor[rgb]{0.56,0.35,0.01}{\textbf{\textit{#1}}}}
\newcommand{\ErrorTok}[1]{\textcolor[rgb]{0.64,0.00,0.00}{\textbf{#1}}}
\newcommand{\ExtensionTok}[1]{#1}
\newcommand{\FloatTok}[1]{\textcolor[rgb]{0.00,0.00,0.81}{#1}}
\newcommand{\FunctionTok}[1]{\textcolor[rgb]{0.13,0.29,0.53}{\textbf{#1}}}
\newcommand{\ImportTok}[1]{#1}
\newcommand{\InformationTok}[1]{\textcolor[rgb]{0.56,0.35,0.01}{\textbf{\textit{#1}}}}
\newcommand{\KeywordTok}[1]{\textcolor[rgb]{0.13,0.29,0.53}{\textbf{#1}}}
\newcommand{\NormalTok}[1]{#1}
\newcommand{\OperatorTok}[1]{\textcolor[rgb]{0.81,0.36,0.00}{\textbf{#1}}}
\newcommand{\OtherTok}[1]{\textcolor[rgb]{0.56,0.35,0.01}{#1}}
\newcommand{\PreprocessorTok}[1]{\textcolor[rgb]{0.56,0.35,0.01}{\textit{#1}}}
\newcommand{\RegionMarkerTok}[1]{#1}
\newcommand{\SpecialCharTok}[1]{\textcolor[rgb]{0.81,0.36,0.00}{\textbf{#1}}}
\newcommand{\SpecialStringTok}[1]{\textcolor[rgb]{0.31,0.60,0.02}{#1}}
\newcommand{\StringTok}[1]{\textcolor[rgb]{0.31,0.60,0.02}{#1}}
\newcommand{\VariableTok}[1]{\textcolor[rgb]{0.00,0.00,0.00}{#1}}
\newcommand{\VerbatimStringTok}[1]{\textcolor[rgb]{0.31,0.60,0.02}{#1}}
\newcommand{\WarningTok}[1]{\textcolor[rgb]{0.56,0.35,0.01}{\textbf{\textit{#1}}}}
\usepackage{longtable,booktabs,array}
\usepackage{calc} % for calculating minipage widths
% Correct order of tables after \paragraph or \subparagraph
\usepackage{etoolbox}
\makeatletter
\patchcmd\longtable{\par}{\if@noskipsec\mbox{}\fi\par}{}{}
\makeatother
% Allow footnotes in longtable head/foot
\IfFileExists{footnotehyper.sty}{\usepackage{footnotehyper}}{\usepackage{footnote}}
\makesavenoteenv{longtable}
\usepackage{graphicx}
\makeatletter
\def\maxwidth{\ifdim\Gin@nat@width>\linewidth\linewidth\else\Gin@nat@width\fi}
\def\maxheight{\ifdim\Gin@nat@height>\textheight\textheight\else\Gin@nat@height\fi}
\makeatother
% Scale images if necessary, so that they will not overflow the page
% margins by default, and it is still possible to overwrite the defaults
% using explicit options in \includegraphics[width, height, ...]{}
\setkeys{Gin}{width=\maxwidth,height=\maxheight,keepaspectratio}
% Set default figure placement to htbp
\makeatletter
\def\fps@figure{htbp}
\makeatother
\setlength{\emergencystretch}{3em} % prevent overfull lines
\providecommand{\tightlist}{%
  \setlength{\itemsep}{0pt}\setlength{\parskip}{0pt}}
\setcounter{secnumdepth}{-\maxdimen} % remove section numbering
\ifLuaTeX
  \usepackage{selnolig}  % disable illegal ligatures
\fi
\usepackage{bookmark}
\IfFileExists{xurl.sty}{\usepackage{xurl}}{} % add URL line breaks if available
\urlstyle{same}
\hypersetup{
  pdftitle={Data Summary and EDA Report},
  pdfauthor={Atalor Polycarp Ehiz},
  hidelinks,
  pdfcreator={LaTeX via pandoc}}

\title{Data Summary and EDA Report}
\author{Atalor Polycarp Ehiz}
\date{2024-11-27}

\begin{document}
\maketitle

\section{** Import Data}\label{import-data}

\begin{Shaded}
\begin{Highlighting}[]
\NormalTok{     my\_data }\OtherTok{\textless{}{-}} \FunctionTok{read.csv}\NormalTok{(}\StringTok{"brain{-}cancer{-}dataset.csv"}\NormalTok{)}
\end{Highlighting}
\end{Shaded}

\section{\texorpdfstring{\textbf{1.
Introduction}}{1. Introduction}}\label{introduction}

This report analyzes clinical and genetic data consisting of mostly
children with brain cancer. The goal is to prepare the dataset for
machine learning models that predict:

\begin{itemize}
\item
  \textbf{OS.Status (Overall Survival Status):} Whether a patient
  survived.
\item
  \textbf{DFS.Status (Disease-Free Survival Status):} Whether a patient
  remained disease-free.
\end{itemize}

The analysis includes:

Preprocessing: Cleaning the data to ensure quality. Feature Selection:
Identifying relevant clinical and genetic variables. Exploratory Data
Analysis (EDA): Understanding relationships between variables.

\section{\texorpdfstring{\textbf{2. Description of the
Dataset}}{2. Description of the Dataset}}\label{description-of-the-dataset}

\subsubsection{\texorpdfstring{2.1\textbf{Dataset
Overview}}{2.1Dataset Overview}}\label{dataset-overview}

\begin{itemize}
\tightlist
\item
  \textbf{Number of Records:} 218
\item
  \textbf{Number of Features:} 61
\item
  \textbf{Missing Values:} Yes (handled during preprocessing)
\end{itemize}

\subsubsection{\texorpdfstring{2.2 \textbf{Variable
Types}}{2.2 Variable Types}}\label{variable-types}

\begin{itemize}
\tightlist
\item
  \textbf{Categorical Variables:} Features like Cancer.Type, Treatment,
  and Tumor.Type.
\item
  \textbf{Numerical Variables:} Age-related data, mutation counts, and
  other clinical measurements.
\end{itemize}

\section{\texorpdfstring{\textbf{3. Preprocessing
Steps}}{3. Preprocessing Steps}}\label{preprocessing-steps}

To ensure the dataset is clean and ready for analysis, several
preprocessing steps were undertaken.

\subsubsection{\texorpdfstring{\textbf{3.1 Handling Missing
Values}}{3.1 Handling Missing Values}}\label{handling-missing-values}

\begin{itemize}
\item
  \textbf{Numerical Variables:} Missing values were replaced with the
  median to preserve the data's distribution.
\item
  \textbf{Categorical Variables:} Missing values were replaced with
  ``Unknown'' for consistency.
\end{itemize}

\begin{Shaded}
\begin{Highlighting}[]
\CommentTok{\# Handle Missing Values}
\NormalTok{my\_data }\OtherTok{\textless{}{-}}\NormalTok{ my\_data }\SpecialCharTok{\%\textgreater{}\%}
  \FunctionTok{mutate}\NormalTok{(}\FunctionTok{across}\NormalTok{(}\FunctionTok{where}\NormalTok{(is.numeric), }\SpecialCharTok{\textasciitilde{}} \FunctionTok{ifelse}\NormalTok{(}\FunctionTok{is.na}\NormalTok{(.), }\FunctionTok{median}\NormalTok{(., }\AttributeTok{na.rm =} \ConstantTok{TRUE}\NormalTok{), .))) }\SpecialCharTok{\%\textgreater{}\%}
  \FunctionTok{mutate}\NormalTok{(}\FunctionTok{across}\NormalTok{(}\FunctionTok{where}\NormalTok{(is.character), }\SpecialCharTok{\textasciitilde{}} \FunctionTok{ifelse}\NormalTok{(}\FunctionTok{is.na}\NormalTok{(.), }\StringTok{"Unknown"}\NormalTok{, .))) }\SpecialCharTok{\%\textgreater{}\%}
  \FunctionTok{mutate}\NormalTok{(}\FunctionTok{across}\NormalTok{(}\FunctionTok{where}\NormalTok{(is.factor), }\SpecialCharTok{\textasciitilde{}} \FunctionTok{ifelse}\NormalTok{(}\FunctionTok{is.na}\NormalTok{(.), }\FunctionTok{as.character}\NormalTok{(}\FunctionTok{levels}\NormalTok{(.)[}\DecValTok{1}\NormalTok{]), .)))}

\CommentTok{\# Verify no missing values remain}
\ControlFlowTok{if}\NormalTok{ (}\FunctionTok{sum}\NormalTok{(}\FunctionTok{is.na}\NormalTok{(my\_data)) }\SpecialCharTok{\textgreater{}} \DecValTok{0}\NormalTok{) \{}
  \FunctionTok{stop}\NormalTok{(}\StringTok{"Unresolved missing values in the dataset."}\NormalTok{)}
\NormalTok{\}}
\end{Highlighting}
\end{Shaded}

\subsubsection{\texorpdfstring{\textbf{3.2 Feature
Cleaning}}{3.2 Feature Cleaning}}\label{feature-cleaning}

\begin{itemize}
\item
  Column names were trimmed and standardized for compatibility with R
  functions.
\item
  All categorical variables were converted into factors.
\end{itemize}

\begin{Shaded}
\begin{Highlighting}[]
\CommentTok{\# Standardize Column Names}
\FunctionTok{colnames}\NormalTok{(my\_data) }\OtherTok{\textless{}{-}} \FunctionTok{trimws}\NormalTok{(}\FunctionTok{colnames}\NormalTok{(my\_data))}
\FunctionTok{colnames}\NormalTok{(my\_data) }\OtherTok{\textless{}{-}} \FunctionTok{make.names}\NormalTok{(}\FunctionTok{colnames}\NormalTok{(my\_data), }\AttributeTok{unique =} \ConstantTok{TRUE}\NormalTok{)}

\CommentTok{\# Convert Categorical Variables to Factors}
\NormalTok{my\_data }\OtherTok{\textless{}{-}}\NormalTok{ my\_data }\SpecialCharTok{\%\textgreater{}\%} \FunctionTok{mutate}\NormalTok{(}\FunctionTok{across}\NormalTok{(}\FunctionTok{where}\NormalTok{(is.character), as.factor))}
\end{Highlighting}
\end{Shaded}

\subsubsection{\texorpdfstring{\textbf{3.3 Feature
Filtering}}{3.3 Feature Filtering}}\label{feature-filtering}

\begin{itemize}
\item
  \textbf{Variance Filtering:} Features with low variance
  (\textless0.01) were removed as they add little information.
\item
  \textbf{Correlation Filtering:} Features with high correlation
  (\textgreater0.8) were removed to reduce redundancy.
\end{itemize}

\begin{Shaded}
\begin{Highlighting}[]
\CommentTok{\# Filter Low{-}Variance Features}
\NormalTok{variances }\OtherTok{\textless{}{-}} \FunctionTok{apply}\NormalTok{(my\_data[, }\FunctionTok{sapply}\NormalTok{(my\_data, is.numeric)], }\DecValTok{2}\NormalTok{, var, }\AttributeTok{na.rm =} \ConstantTok{TRUE}\NormalTok{)}
\NormalTok{data\_filtered }\OtherTok{\textless{}{-}}\NormalTok{ my\_data[, variances }\SpecialCharTok{\textgreater{}} \FloatTok{0.01}\NormalTok{]}

\CommentTok{\# Remove Highly Correlated Features}
\NormalTok{cor\_matrix }\OtherTok{\textless{}{-}} \FunctionTok{cor}\NormalTok{(data\_filtered[, }\FunctionTok{sapply}\NormalTok{(data\_filtered, is.numeric)], }\AttributeTok{use =} \StringTok{"complete.obs"}\NormalTok{)}
\NormalTok{high\_corr }\OtherTok{\textless{}{-}} \FunctionTok{findCorrelation}\NormalTok{(cor\_matrix, }\AttributeTok{cutoff =} \FloatTok{0.8}\NormalTok{)}
\NormalTok{data\_filtered }\OtherTok{\textless{}{-}}\NormalTok{ data\_filtered[, }\SpecialCharTok{{-}}\NormalTok{high\_corr]}
\end{Highlighting}
\end{Shaded}

\subsubsection{\texorpdfstring{\textbf{3.4 Feature
Selection}}{3.4 Feature Selection}}\label{feature-selection}

\begin{Shaded}
\begin{Highlighting}[]
   \CommentTok{\# Define categorical columns}
\NormalTok{categorical\_columns }\OtherTok{\textless{}{-}} \FunctionTok{c}\NormalTok{(}
  \StringTok{"Study.ID"}\NormalTok{, }\StringTok{"Patient.ID"}\NormalTok{, }\StringTok{"Sample.ID"}\NormalTok{, }\StringTok{"Age.Class"}\NormalTok{, }\StringTok{"BRAF\_RELA.Status"}\NormalTok{, }
  \StringTok{"BRAF.Status"}\NormalTok{, }\StringTok{"BRAF.Status2"}\NormalTok{, }\StringTok{"Cancer.Predispositions"}\NormalTok{, }\StringTok{"Cancer.Type"}\NormalTok{, }
  \StringTok{"Cancer.Type.Detailed"}\NormalTok{, }\StringTok{"Chemotherapy"}\NormalTok{, }\StringTok{"Chemotherapy.Agents"}\NormalTok{, }
  \StringTok{"Chemotherapy.Type"}\NormalTok{, }\StringTok{"Clinical.Status.at.Collection.Event"}\NormalTok{, }
  \StringTok{"CTNNB1.Status"}\NormalTok{, }\StringTok{"DFS.Status"}\NormalTok{, }\StringTok{"Ethnicity"}\NormalTok{, }\StringTok{"Extent.of.Tumor.Resection"}\NormalTok{, }
  \StringTok{"External.Patient.ID"}\NormalTok{, }\StringTok{"Formulation"}\NormalTok{, }\StringTok{"H3F3A\_CTNNB1.Status"}\NormalTok{, }
  \StringTok{"Initial.CNS.Tumor.Diagnosis.Related.to.OS"}\NormalTok{, }\StringTok{"Initial.Diagnosis.Type"}\NormalTok{, }
  \StringTok{"LGG\_BRAF.Status"}\NormalTok{, }\StringTok{"Medical.Conditions"}\NormalTok{, }\StringTok{"Multiple.Cancer.Predispositions"}\NormalTok{, }
  \StringTok{"Multiple.Medical.Conditions"}\NormalTok{, }\StringTok{"Multiple.Tumor.Locations"}\NormalTok{, }\StringTok{"Oncotree.Code"}\NormalTok{, }
  \StringTok{"OS.Status"}\NormalTok{, }\StringTok{"Protocol.and.Treatment.Arm"}\NormalTok{, }\StringTok{"Race"}\NormalTok{, }\StringTok{"Radiation"}\NormalTok{, }
  \StringTok{"Radiation.Site"}\NormalTok{, }\StringTok{"Radiation.Type"}\NormalTok{, }\StringTok{"Sample.Annotation"}\NormalTok{, }\StringTok{"Sample.Origin"}\NormalTok{, }
  \StringTok{"Sex"}\NormalTok{, }\StringTok{"Surgery"}\NormalTok{, }\StringTok{"Treatment"}\NormalTok{, }\StringTok{"Treatment.Changed"}\NormalTok{, }\StringTok{"Treatment.Status"}\NormalTok{, }
  \StringTok{"Tumor.Location.Condensed"}\NormalTok{, }\StringTok{"Tumor.Tissue.Site"}\NormalTok{, }\StringTok{"Tumor.Type"}\NormalTok{, }\StringTok{"Updated.Grade"}
\NormalTok{)}
\end{Highlighting}
\end{Shaded}

Significant features were identified for both OS.Status and DFS.Status
using a Chi-square test.

\begin{Shaded}
\begin{Highlighting}[]
\CommentTok{\# Chi{-}Square Test for Feature Selection}
\NormalTok{chi\_sq\_results\_OS }\OtherTok{\textless{}{-}} \FunctionTok{lapply}\NormalTok{(categorical\_columns, }\ControlFlowTok{function}\NormalTok{(col) \{}
\NormalTok{  table\_data }\OtherTok{\textless{}{-}} \FunctionTok{table}\NormalTok{(my\_data[[col]], my\_data}\SpecialCharTok{$}\NormalTok{OS.Status)}
  \ControlFlowTok{if}\NormalTok{ (}\FunctionTok{any}\NormalTok{(}\FunctionTok{dim}\NormalTok{(table\_data) }\SpecialCharTok{==} \DecValTok{0}\NormalTok{)) \{}
    \FunctionTok{return}\NormalTok{(}\FunctionTok{data.frame}\NormalTok{(}\AttributeTok{Feature =}\NormalTok{ col, }\AttributeTok{p\_value =} \ConstantTok{NA}\NormalTok{))}
\NormalTok{  \}}
\NormalTok{  test\_result }\OtherTok{\textless{}{-}} \FunctionTok{chisq.test}\NormalTok{(table\_data)}
  \FunctionTok{data.frame}\NormalTok{(}\AttributeTok{Feature =}\NormalTok{ col, }\AttributeTok{p\_value =}\NormalTok{ test\_result}\SpecialCharTok{$}\NormalTok{p.value)}
\NormalTok{\})}

\NormalTok{chi\_sq\_summary\_OS }\OtherTok{\textless{}{-}} \FunctionTok{do.call}\NormalTok{(rbind, chi\_sq\_results\_OS)}
\NormalTok{significant\_features\_OS }\OtherTok{\textless{}{-}}\NormalTok{ chi\_sq\_summary\_OS }\SpecialCharTok{\%\textgreater{}\%} \FunctionTok{filter}\NormalTok{(p\_value }\SpecialCharTok{\textless{}} \FloatTok{0.05}\NormalTok{)}
\end{Highlighting}
\end{Shaded}

\section{4. Key Insights from EDA}\label{key-insights-from-eda}

\subsubsection{\texorpdfstring{\textbf{4.1 Target Variable
Distribution}}{4.1 Target Variable Distribution}}\label{target-variable-distribution}

The target variables (OS.Status and DFS.Status) show balanced
distributions suitable for machine learning.

\begin{Shaded}
\begin{Highlighting}[]
\CommentTok{\# OS.Status Distribution}
\FunctionTok{ggplot}\NormalTok{(my\_data, }\FunctionTok{aes}\NormalTok{(}\AttributeTok{x =}\NormalTok{ OS.Status, }\AttributeTok{fill =}\NormalTok{ OS.Status)) }\SpecialCharTok{+}
  \FunctionTok{geom\_bar}\NormalTok{() }\SpecialCharTok{+}
  \FunctionTok{labs}\NormalTok{(}\AttributeTok{title =} \StringTok{"Distribution of OS.Status"}\NormalTok{, }\AttributeTok{x =} \StringTok{"OS.Status"}\NormalTok{, }\AttributeTok{y =} \StringTok{"Count"}\NormalTok{) }\SpecialCharTok{+}
  \FunctionTok{theme\_minimal}\NormalTok{()}
\end{Highlighting}
\end{Shaded}

\includegraphics{Data-Summary-and-EDA-report_files/figure-latex/unnamed-chunk-7-1.pdf}

\begin{Shaded}
\begin{Highlighting}[]
\CommentTok{\# DFS.Status Distribution}
\FunctionTok{ggplot}\NormalTok{(my\_data, }\FunctionTok{aes}\NormalTok{(}\AttributeTok{x =}\NormalTok{ DFS.Status, }\AttributeTok{fill =}\NormalTok{ DFS.Status)) }\SpecialCharTok{+}
  \FunctionTok{geom\_bar}\NormalTok{() }\SpecialCharTok{+}
  \FunctionTok{labs}\NormalTok{(}\AttributeTok{title =} \StringTok{"Distribution of DFS.Status"}\NormalTok{, }\AttributeTok{x =} \StringTok{"DFS.Status"}\NormalTok{, }\AttributeTok{y =} \StringTok{"Count"}\NormalTok{) }\SpecialCharTok{+}
  \FunctionTok{theme\_minimal}\NormalTok{()}
\end{Highlighting}
\end{Shaded}

\includegraphics{Data-Summary-and-EDA-report_files/figure-latex/unnamed-chunk-7-2.pdf}

\begin{itemize}
\item
  \textbf{Insight:} The OS.Status distribution indicates a balanced
  dataset for survival analysis.
\item
  \textbf{Insight:} The DFS.Status distribution shows similar
  proportions for patients who remained disease-free and those who
  relapsed.
\end{itemize}

\subsubsection{\texorpdfstring{\emph{4.2 Correlation
Analysis}}{4.2 Correlation Analysis}}\label{correlation-analysis}

A heatmap was used to visualize correlations among numerical variables.
Highly correlated features were excluded.

\begin{Shaded}
\begin{Highlighting}[]
\CommentTok{\# Correlation Matrix Visualization}
\FunctionTok{corrplot}\NormalTok{(cor\_matrix, }\AttributeTok{method =} \StringTok{"color"}\NormalTok{, }\AttributeTok{tl.cex =} \FloatTok{0.8}\NormalTok{, }\AttributeTok{number.cex =} \FloatTok{0.7}\NormalTok{, }\AttributeTok{addCoef.col =} \StringTok{"black"}\NormalTok{)}
\end{Highlighting}
\end{Shaded}

\includegraphics{Data-Summary-and-EDA-report_files/figure-latex/unnamed-chunk-8-1.pdf}

\begin{itemize}
\tightlist
\item
  \textbf{Insight:} Some numerical features showed high correlations and
  were excluded to prevent multicollinearity.
\end{itemize}

\subsubsection{4.3 Selected Features}\label{selected-features}

To summarize the features selected for OS.Status and DFS.Status, we
present: - A Bar Plot showing the total number of features selected for
each target variable. - A Table listing the features categorized as
unique to each target variable or common to both.

\begin{itemize}
\tightlist
\item
  \textbf{Features for OS.Status \& DFS.Status}
\end{itemize}

\begin{Shaded}
\begin{Highlighting}[]
\CommentTok{\# Chi{-}square test for OS.Status and DFS.Status}
\NormalTok{chi\_sq\_results\_OS }\OtherTok{\textless{}{-}} \FunctionTok{lapply}\NormalTok{(categorical\_columns, }\ControlFlowTok{function}\NormalTok{(col) \{}
\NormalTok{  table\_data }\OtherTok{\textless{}{-}} \FunctionTok{table}\NormalTok{(my\_data[[col]], my\_data}\SpecialCharTok{$}\NormalTok{OS.Status)}
  \ControlFlowTok{if}\NormalTok{ (}\FunctionTok{any}\NormalTok{(}\FunctionTok{dim}\NormalTok{(table\_data) }\SpecialCharTok{==} \DecValTok{0}\NormalTok{)) \{}
    \FunctionTok{return}\NormalTok{(}\FunctionTok{data.frame}\NormalTok{(}\AttributeTok{Feature =}\NormalTok{ col, }\AttributeTok{p\_value =} \ConstantTok{NA}\NormalTok{))}
\NormalTok{  \}}
\NormalTok{  test\_result }\OtherTok{\textless{}{-}} \FunctionTok{chisq.test}\NormalTok{(table\_data)}
  \FunctionTok{data.frame}\NormalTok{(}\AttributeTok{Feature =}\NormalTok{ col, }\AttributeTok{p\_value =}\NormalTok{ test\_result}\SpecialCharTok{$}\NormalTok{p.value)}
\NormalTok{\})}

\NormalTok{chi\_sq\_summary\_OS }\OtherTok{\textless{}{-}} \FunctionTok{do.call}\NormalTok{(rbind, chi\_sq\_results\_OS)}

\NormalTok{chi\_sq\_results\_DFS }\OtherTok{\textless{}{-}} \FunctionTok{lapply}\NormalTok{(categorical\_columns, }\ControlFlowTok{function}\NormalTok{(col) \{}
\NormalTok{  table\_data }\OtherTok{\textless{}{-}} \FunctionTok{table}\NormalTok{(my\_data[[col]], my\_data}\SpecialCharTok{$}\NormalTok{DFS.Status)}
  \ControlFlowTok{if}\NormalTok{ (}\FunctionTok{any}\NormalTok{(}\FunctionTok{dim}\NormalTok{(table\_data) }\SpecialCharTok{==} \DecValTok{0}\NormalTok{)) \{}
    \FunctionTok{return}\NormalTok{(}\FunctionTok{data.frame}\NormalTok{(}\AttributeTok{Feature =}\NormalTok{ col, }\AttributeTok{p\_value =} \ConstantTok{NA}\NormalTok{))}
\NormalTok{  \}}
\NormalTok{  test\_result }\OtherTok{\textless{}{-}} \FunctionTok{chisq.test}\NormalTok{(table\_data)}
  \FunctionTok{data.frame}\NormalTok{(}\AttributeTok{Feature =}\NormalTok{ col, }\AttributeTok{p\_value =}\NormalTok{ test\_result}\SpecialCharTok{$}\NormalTok{p.value)}
\NormalTok{\})}

\NormalTok{chi\_sq\_summary\_DFS }\OtherTok{\textless{}{-}} \FunctionTok{do.call}\NormalTok{(rbind, chi\_sq\_results\_DFS)}
\end{Highlighting}
\end{Shaded}

\begin{Shaded}
\begin{Highlighting}[]
\CommentTok{\# Filter significant features for OS.Status and DFS.Status}
\NormalTok{significant\_features\_OS }\OtherTok{\textless{}{-}}\NormalTok{ chi\_sq\_summary\_OS }\SpecialCharTok{\%\textgreater{}\%} \FunctionTok{filter}\NormalTok{(p\_value }\SpecialCharTok{\textless{}} \FloatTok{0.05}\NormalTok{)}
\NormalTok{significant\_features\_DFS }\OtherTok{\textless{}{-}}\NormalTok{ chi\_sq\_summary\_DFS }\SpecialCharTok{\%\textgreater{}\%} \FunctionTok{filter}\NormalTok{(p\_value }\SpecialCharTok{\textless{}} \FloatTok{0.05}\NormalTok{)}

\CommentTok{\# Subset data for significant features}
\NormalTok{data\_OS }\OtherTok{\textless{}{-}}\NormalTok{ my\_data }\SpecialCharTok{\%\textgreater{}\%} \FunctionTok{select}\NormalTok{(}\FunctionTok{all\_of}\NormalTok{(significant\_features\_OS}\SpecialCharTok{$}\NormalTok{Feature), OS.Status)}
\NormalTok{data\_DFS }\OtherTok{\textless{}{-}}\NormalTok{ my\_data }\SpecialCharTok{\%\textgreater{}\%} \FunctionTok{select}\NormalTok{(}\FunctionTok{all\_of}\NormalTok{(significant\_features\_DFS}\SpecialCharTok{$}\NormalTok{Feature), DFS.Status)}
\end{Highlighting}
\end{Shaded}

\begin{Shaded}
\begin{Highlighting}[]
\CommentTok{\# Extract selected features}
\NormalTok{selected\_features\_OS }\OtherTok{\textless{}{-}} \FunctionTok{colnames}\NormalTok{(data\_OS)[}\FunctionTok{colnames}\NormalTok{(data\_OS) }\SpecialCharTok{!=} \StringTok{"OS.Status"}\NormalTok{]}
\NormalTok{selected\_features\_DFS }\OtherTok{\textless{}{-}} \FunctionTok{colnames}\NormalTok{(data\_DFS)[}\FunctionTok{colnames}\NormalTok{(data\_DFS) }\SpecialCharTok{!=} \StringTok{"DFS.Status"}\NormalTok{]}

\CommentTok{\# Create a summary of feature counts}
\NormalTok{feature\_counts }\OtherTok{\textless{}{-}} \FunctionTok{data.frame}\NormalTok{(}
  \AttributeTok{Target =} \FunctionTok{c}\NormalTok{(}\StringTok{"OS.Status"}\NormalTok{, }\StringTok{"DFS.Status"}\NormalTok{),}
  \AttributeTok{Features =} \FunctionTok{c}\NormalTok{(}\FunctionTok{length}\NormalTok{(selected\_features\_OS), }\FunctionTok{length}\NormalTok{(selected\_features\_DFS)),}
  \AttributeTok{Common =} \FunctionTok{length}\NormalTok{(}\FunctionTok{intersect}\NormalTok{(selected\_features\_OS, selected\_features\_DFS))}
\NormalTok{)}
\end{Highlighting}
\end{Shaded}

\begin{Shaded}
\begin{Highlighting}[]
\CommentTok{\# Create a bar plot}
\FunctionTok{ggplot}\NormalTok{(feature\_counts, }\FunctionTok{aes}\NormalTok{(}\AttributeTok{x =}\NormalTok{ Target, }\AttributeTok{y =}\NormalTok{ Features, }\AttributeTok{fill =}\NormalTok{ Target)) }\SpecialCharTok{+}
  \FunctionTok{geom\_bar}\NormalTok{(}\AttributeTok{stat =} \StringTok{"identity"}\NormalTok{, }\AttributeTok{position =} \StringTok{"dodge"}\NormalTok{, }\AttributeTok{color =} \StringTok{"black"}\NormalTok{) }\SpecialCharTok{+}
  \FunctionTok{geom\_text}\NormalTok{(}\FunctionTok{aes}\NormalTok{(}\AttributeTok{label =}\NormalTok{ Features), }\AttributeTok{vjust =} \SpecialCharTok{{-}}\FloatTok{0.5}\NormalTok{, }\AttributeTok{size =} \DecValTok{4}\NormalTok{) }\SpecialCharTok{+}
  \FunctionTok{labs}\NormalTok{(}
    \AttributeTok{title =} \StringTok{"Number of Features Selected for OS.Status and DFS.Status"}\NormalTok{,}
    \AttributeTok{x =} \StringTok{"Target Variable"}\NormalTok{,}
    \AttributeTok{y =} \StringTok{"Number of Features"}
\NormalTok{  ) }\SpecialCharTok{+}
  \FunctionTok{theme\_minimal}\NormalTok{() }\SpecialCharTok{+}
  \FunctionTok{scale\_fill\_brewer}\NormalTok{(}\AttributeTok{palette =} \StringTok{"Set2"}\NormalTok{)}
\end{Highlighting}
\end{Shaded}

\includegraphics{Data-Summary-and-EDA-report_files/figure-latex/unnamed-chunk-12-1.pdf}

\begin{itemize}
\tightlist
\item
  \textbf{Union of Features:}
\end{itemize}

\begin{Shaded}
\begin{Highlighting}[]
\CommentTok{\# Categorize features}
\NormalTok{feature\_categories }\OtherTok{\textless{}{-}} \FunctionTok{data.frame}\NormalTok{(}
  \AttributeTok{Feature =} \FunctionTok{union}\NormalTok{(selected\_features\_OS, selected\_features\_DFS),}
  \AttributeTok{Category =} \FunctionTok{ifelse}\NormalTok{(}
    \FunctionTok{union}\NormalTok{(selected\_features\_OS, selected\_features\_DFS) }\SpecialCharTok{\%in\%} \FunctionTok{intersect}\NormalTok{(selected\_features\_OS, selected\_features\_DFS), }\StringTok{"Both"}\NormalTok{,}
    \FunctionTok{ifelse}\NormalTok{(}\FunctionTok{union}\NormalTok{(selected\_features\_OS, selected\_features\_DFS) }\SpecialCharTok{\%in\%}\NormalTok{ selected\_features\_OS, }\StringTok{"OS.Status"}\NormalTok{, }\StringTok{"DFS.Status"}\NormalTok{)}
\NormalTok{  )}
\NormalTok{)}

\CommentTok{\# Display the table}
\NormalTok{knitr}\SpecialCharTok{::}\FunctionTok{kable}\NormalTok{(}
\NormalTok{  feature\_categories,}
  \AttributeTok{caption =} \StringTok{"Feature Categories by Target Variable"}\NormalTok{,}
  \AttributeTok{col.names =} \FunctionTok{c}\NormalTok{(}\StringTok{"Feature Name"}\NormalTok{, }\StringTok{"Category"}\NormalTok{)}
\NormalTok{)}
\end{Highlighting}
\end{Shaded}

\begin{longtable}[]{@{}ll@{}}
\caption{Feature Categories by Target Variable}\tabularnewline
\toprule\noalign{}
Feature Name & Category \\
\midrule\noalign{}
\endfirsthead
\toprule\noalign{}
Feature Name & Category \\
\midrule\noalign{}
\endhead
\bottomrule\noalign{}
\endlastfoot
Study.ID & OS.Status \\
Age.Class & OS.Status \\
BRAF\_RELA.Status & Both \\
BRAF.Status & Both \\
BRAF.Status2 & Both \\
Cancer.Predispositions & Both \\
Cancer.Type & Both \\
Cancer.Type.Detailed & Both \\
Chemotherapy & Both \\
Chemotherapy.Agents & OS.Status \\
Chemotherapy.Type & Both \\
Clinical.Status.at.Collection.Event & Both \\
DFS.Status & OS.Status \\
Extent.of.Tumor.Resection & Both \\
Formulation & OS.Status \\
H3F3A\_CTNNB1.Status & Both \\
Initial.CNS.Tumor.Diagnosis.Related.to.OS & Both \\
Initial.Diagnosis.Type & Both \\
LGG\_BRAF.Status & Both \\
Medical.Conditions & OS.Status \\
Multiple.Cancer.Predispositions & Both \\
Multiple.Medical.Conditions & OS.Status \\
Multiple.Tumor.Locations & Both \\
Oncotree.Code & Both \\
Radiation & Both \\
Sample.Annotation & Both \\
Sample.Origin & Both \\
Surgery & Both \\
Treatment & Both \\
Treatment.Status & Both \\
Tumor.Location.Condensed & Both \\
Tumor.Tissue.Site & OS.Status \\
Tumor.Type & Both \\
Updated.Grade & Both \\
OS.Status & DFS.Status \\
Race & DFS.Status \\
Radiation.Type & DFS.Status \\
Treatment.Changed & DFS.Status \\
\end{longtable}

\section{Summary}\label{summary}

\begin{itemize}
\tightlist
\item
  Dataset cleaning and preprocessing ensured high data quality.
\item
  Significant features were selected for predictive modeling.
\item
  EDA revealed balanced target variable distributions and relationships
  between features.
\end{itemize}

\subsection{End}\label{end}

\end{document}
